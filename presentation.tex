\documentclass{beamer}

% packages
\usepackage{graphicx}
  \graphicspath{{figures}}
\usepackage{minted}
\usepackage{amssymb}

% subfig support
\usepackage{caption}
\usepackage{subcaption}

% packages
\usepackage{biblatex}
  \addbibresource{bibliography.bib}
\usepackage[acronym,nomain]{glossaries}
  \setacronymstyle{long-short}
  \newacronym[shortplural={DoFs},longplural={degrees-of-freedom}]
    {dof}{DoF}{degree-of-freedom}
  \newacronym{fem}{FEM}{the finite element method}
  \newacronym{pde}{PDE}{partial differential equation}
  \newacronym[shortplural={FLOPs},longplural={floating-point operations}]
    {flop}{FLOP}{floating-point operation}
  \newacronym{dag}{DAG}{directed acyclic graph}
  \newacronym{dg}{DG}{discontinous Galerkin}
  \newacronym{poset}{poset}{partially-ordered set}
  \newacronym{rcm}{RCM}{reverse Cuthill-McKee}
  \newacronym{dsl}{DSL}{domain-specific language}
  \newacronym{jit}{JIT}{just-in-time}
  \newacronym{ufl}{UFL}{the Unified Form Language}
  \newacronym{tsfc}{TSFC}{the Two-Stage Form Compiler}
\usepackage{graphicx}
  \graphicspath{{figures}}
\usepackage{minted}
\usepackage{todonotes}
\usepackage{hyperref}
\usepackage{subcaption}
\usepackage{amsmath}
% source: https://tex.stackexchange.com/questions/650034/mathbb-font-for-lowercase-letters
\usepackage[bb=libus]{mathalpha}
\usepackage{pgf}
\usepackage{pgfplots}
\usepackage{tikz}
\usepackage{tkz-euclide}
  \usetikzlibrary{arrows,calc,graphs,graphdrawing,positioning,tikzmark,shapes.geometric,patterns.meta,decorations.pathreplacing}
  \usegdlibrary{trees}
  \pgfdeclarelayer{background}
  \pgfsetlayers{background,main}
  \tikzstyle{ptlabel} = [anchor=center, color=black, opacity=1]
  \tikzset{font={\small}}
  \tikzset{label style/.append style={font=\small}}
  % source: https://tex.stackexchange.com/questions/356564/macro-for-rounded-polygon-around-some-nodes
  \def\drawpolygon#1,#2;{
    \begin{pgfonlayer}{background}
        \filldraw[line width=28,join=round](#1.center)foreach\A in{#2}{--(\A.center)}--cycle;
        \filldraw[line width=27,join=round,blue!10](#1.center)foreach\A in{#2}{--(\A.center)}--cycle;
    \end{pgfonlayer}
  }

% theme info
\usetheme{firedrake}

% title info
\title{\texttt{pyop3}: A new domain-specific language for automating high-performance mesh-based simulation codes}
\author{Connor Ward (supervised by David Ham)}
\date{January 2023}

% macros

% checkbox
% source https://tex.stackexchange.com/questions/16000/creating-boxed-check-mark
\newcommand{\unchecked}{\makebox[0pt][l]{$\square$}\raisebox{.15ex}{\hspace{0.1em}$\quad$}}
\newcommand{\maybe}{\makebox[0pt][l]{$\square$}\raisebox{.15ex}{\hspace{0.1em}$\lozenge$}}
\newcommand{\checked}{\makebox[0pt][l]{$\square$}\raisebox{.15ex}{\hspace{0.1em}$\checkmark$}}

% hacky way to get \pyop2 and \pyop3 as valid macros
% source: https://tex.stackexchange.com/questions/13290/how-to-define-macros-with-numbers-in-them
\def\pyop#1{\ifnum#1=2 {PyOP2}\else \ifnum#1=3 {\texttt{pyop3}}\fi \fi}

\newcommand{\basichasse}{%
  \begin{scope}[auto,every node/.style={circle,minimum size=20pt,draw,color=black,fill=white}]
    \begin{scope}[yshift=0cm]
      \node (1) [xshift={1*\textwidth/3}] {1};
      \node (2) [xshift={2*\textwidth/3}] {2};
    \end{scope}

    \begin{scope}[yshift=2cm]
      \node (7) [xshift={1*\textwidth/6}] {7};
      \node (8) [xshift={2*\textwidth/6}] {8};
      \node (9) [xshift={3*\textwidth/6}] {9};
      \node (10) [xshift={4*\textwidth/6}] {10};
      \node (11) [xshift={5*\textwidth/6}] {11};
    \end{scope}

    \begin{scope}[yshift=4cm]
      \node (3) [xshift={1*\textwidth/5}] {3};
      \node (4) [xshift={2*\textwidth/5}] {4};
      \node (5) [xshift={3*\textwidth/5}] {5};
      \node (6) [xshift={4*\textwidth/5}] {6};
    \end{scope}

    \draw [-Stealth] (1) -- (7);
    \draw [-Stealth] (1) -- (8);
    \draw [-Stealth] (1) -- (9);
    \draw [-Stealth] (2) -- (9);
    \draw [-Stealth] (2) -- (10);
    \draw [-Stealth] (2) -- (11);
    \draw [-Stealth] (7) -- (3);
    \draw [-Stealth] (7) -- (5);
    \draw [-Stealth] (8) -- (3);
    \draw [-Stealth] (8) -- (4);
    \draw [-Stealth] (9) -- (4);
    \draw [-Stealth] (9) -- (5);
    \draw [-Stealth] (10) -- (4);
    \draw [-Stealth] (10) -- (6);
    \draw [-Stealth] (11) -- (5);
    \draw [-Stealth] (11) -- (6);
  \end{scope}
}

\newcommand{\py}{\mintinline{python}}
\newcommand{\clang}{\mintinline{c}}
\newcommand{\closure}{\mathbb{cl}}
\newcommand{\support}{\mathbb{supp}}
\newcommand{\plexstar}{\mathbb{st}}
\newcommand{\cone}{\mathbb{cone}}

\newcommand{\hdiv}{$H(\mathrm{div})$ }
\newcommand{\hcurl}{$H(\mathrm{curl})$ }

% drawing triangles
\newcommand{\trianglestyles}{%
  \tikzstyle {cellcolor} = [fill=yellow!80];
  \tikzstyle {edgecolor} = [fill=red!60];
  \tikzstyle {vertcolor} = [fill=blue!30];
  \tikzstyle {segment} = [line width=1.2pt];
  \tikzstyle {dof} = [draw=black,line width=1.2pt];
  \tikzstyle {celldof} = [dof,cellcolor];
  \tikzstyle {edgedof} = [dof,edgecolor];
  \tikzstyle {vertdof} = [dof,vertcolor];
  \tikzstyle {doftext} = [font=\bf];
  \tikzstyle {vdof} = [-stealth,draw=red!60,line width=1.9];
}

\newcommand{\stylesegmentsone}{%
  \tkzSetUpStyle[
    postaction=decorate,
    decoration={
      markings,
      mark=at position .53 with {\arrow[very thick]{##1}},
    }
  ]{myarrow}
}

\newcommand{\stylesegmentstwo}{%
  \tkzSetUpStyle[
    postaction=decorate,
    decoration={
      markings,
      mark=at position .78 with {\arrow[very thick]{##1}},
      mark=at position .28 with {\arrow[very thick]{##1}},
    }
  ]{myarrow}
}

\newcommand{\deftriangle}{%
  \trianglestyles
  \tkzDefPoint(0,0){v0}
  \tkzDefShiftPoint[v0](0:2.5){v1}
  \tkzDefShiftPoint[v0](90:2.5){v2}
}

\newcommand{\defsmalltriangle}{%
  \trianglestyles
  \tkzDefPoint(0,0){v0}
  \tkzDefShiftPoint[v0](0:1.5){v1}
  \tkzDefShiftPoint[v0](90:1.5){v2}
}

\newcommand{\defbigtriangle}{%
  \trianglestyles
  \tkzDefPoint(0,0){v0}
  \tkzDefShiftPoint[v0](0:3.8){v1}
  \tkzDefShiftPoint[v0](90:3.8){v2}
}

\newcommand{\drawtriangle}{%
  \deftriangle
  \tkzDrawSegment[myarrow=stealth,segment](v0,v1)
  \tkzDrawSegment[myarrow=stealth,segment](v1,v2)
  \tkzDrawSegment[myarrow=stealth,segment](v0,v2)
}

\newcommand{\drawbigtriangle}{%
  \defbigtriangle
  \tkzDrawSegment[myarrow=stealth,segment](v0,v1)
  \tkzDrawSegment[myarrow=stealth,segment](v1,v2)
  \tkzDrawSegment[myarrow=stealth,segment](v0,v2)
}

\newcommand{\drawsmalltriangle}{%
  \defsmalltriangle
  \tkzDrawSegment[myarrow=stealth,segment](v0,v1)
  \tkzDrawSegment[myarrow=stealth,segment](v1,v2)
  \tkzDrawSegment[myarrow=stealth,segment](v0,v2)
}

\newcommand{\drawflippedtriangle}{%
  \deftriangle
  \tkzDrawSegment[myarrow=stealth,segment](v0,v1)
  \tkzDrawSegment[myarrow=stealth,segment](v2,v1)
  \tkzDrawSegment[myarrow=stealth,segment](v0,v2)
}

\newcommand{\mixedstylesetup}{%
  \tikzstyle{v0} = [fill=blue!50];
  \tikzstyle{v1} = [fill=red!65];
}

\begin{document}

\frame{\titlepage}

\begin{frame}{What came before: \pyop2}
  \begin{itemize}
    \item Domain-specific language embedded in Python for doing mesh computations
    \item Uses code generation to produce fast code
    \item Handles the data structures used by Firedrake
    \item Used all over Firedrake for things like assembly and interpolation
  \end{itemize}
\end{frame}

\begin{frame}{Introducing \pyop3}
  Key differences with \pyop2:
  \begin{itemize}
    \item Complete rewrite of the code generation part
    \item New more expressive and composable interface inspired by PETSc DMPlex
    \item \textbf{Has a new data layout model for describing mesh data}
    \item Work-in-progress!
  \end{itemize}
\end{frame}

\begin{frame}{\pyop3 wishlist}
  We want an abstraction that lets us easily express and automate the following.

  \textbf{Features:}

  \unchecked Handle orientations (e.g. unstructured hexes) \\
  \unchecked p-adaptivity \\
  \unchecked Mixed meshes \\

  \textbf{Performance:}

  \unchecked Exploit mesh partial structure (e.g. extruded) \\
  \unchecked Prescribe DoF ordering (e.g. in extruded mesh columns) \\

  (And more\dots)
\end{frame}

\begin{frame}
  Claim: \pyop3's new mesh data layout abstraction facilitates all of these. 
\end{frame}

\begin{frame}{How does \pyop2 store mesh data?}
  \begin{itemize}
    \item Mesh data is stored by \py{Dats}\footnote{Sparse matrices not discussed}
    \item \py{Mixed Dats} and \py{Dats} for extruded meshes are also possible
    \item These associate a fixed inner shape ($d_m$) with a set of possibly unordered nodes ($i_n$)
    \item \textbf{This abstraction loses topological information}
  \end{itemize}


  \begin{columns}
    \begin{column}{.5\textwidth}
      \centering
      \begin{tikzpicture}
        \stylesegmentsone
        \drawbigtriangle

        \filldraw [vertdof] (v0) ++ (.2,0) node [doftext] {$i_0^1$} circle [radius=8pt];
        \filldraw [vertdof] (v0) ++ (-.25,.2) node [doftext] {$i_0^0$} circle [radius=8pt];

        \filldraw [vertdof] (v1) ++ (.2,0) node [doftext] {$i_1^1$} circle [radius=8pt];
        \filldraw [vertdof] (v1) ++ (-.25,.2) node [doftext] {$i_1^0$} circle [radius=8pt];

        \filldraw [vertdof] (v2) ++ (.2,0) node [doftext] {$i_2^1$} circle [radius=8pt];
        \filldraw [vertdof] (v2) ++ (-.25,.2) node [doftext] {$i_2^0$} circle [radius=8pt];

        % edge dofs
        \tkzDefBarycentricPoint(v0=2.3,v1=1) \tkzGetPoint{e0d0}
        \filldraw [vertdof] (e0d0) ++ (.2,0) node [doftext] {$i_7^1$} circle [radius=8pt];
        \filldraw [vertdof] (e0d0) ++ (-.25,.2) node [doftext] {$i_7^0$} circle [radius=8pt];

        \tkzDefBarycentricPoint(v0=1,v1=2.3) \tkzGetPoint{e0d1}
        \filldraw [vertdof] (e0d1) ++ (.2,0) node [doftext] {$i_8^1$} circle [radius=8pt];
        \filldraw [vertdof] (e0d1) ++ (-.25,.2) node [doftext] {$i_8^0$} circle [radius=8pt];

        \tkzDefBarycentricPoint(v1=2.3,v2=1) \tkzGetPoint{e1d0}
        \filldraw [vertdof] (e1d0) ++ (.2,0) node [doftext] {$i_3^1$} circle [radius=8pt];
        \filldraw [vertdof] (e1d0) ++ (-.25,.2) node [doftext] {$i_3^0$} circle [radius=8pt];

        \tkzDefBarycentricPoint(v1=1,v2=2.3) \tkzGetPoint{e1d1}
        \filldraw [vertdof] (e1d1) ++ (.2,0) node [doftext] {$i_4^1$} circle [radius=8pt];
        \filldraw [vertdof] (e1d1) ++ (-.25,.2) node [doftext] {$i_4^0$} circle [radius=8pt];

        \tkzDefBarycentricPoint(v0=2.3,v2=1) \tkzGetPoint{e2d0}
        \filldraw [vertdof] (e2d0) ++ (.2,0) node [doftext] {$i_5^1$} circle [radius=8pt];
        \filldraw [vertdof] (e2d0) ++ (-.25,.2) node [doftext] {$i_5^0$} circle [radius=8pt];

        \tkzDefBarycentricPoint(v0=1,v2=2.3) \tkzGetPoint{e2d1}
        \filldraw [vertdof] (e2d1) ++ (.2,0) node [doftext] {$i_6^1$} circle [radius=8pt];
        \filldraw [vertdof] (e2d1) ++ (-.25,.2) node [doftext] {$i_6^0$} circle [radius=8pt];

        % cell dof
        \tkzDefBarycentricPoint(v0=1,v1=1,v2=1) \tkzGetPoint{c0d0}
        \filldraw [vertdof] (c0d0) ++ (.2,0) node [doftext] {$i_9^1$} circle [radius=8pt];
        \filldraw [vertdof] (c0d0) ++ (-.25,.2) node [doftext] {$i_9^0$} circle [radius=8pt];
      \end{tikzpicture}
    \end{column}

    \begin{column}{.5\textwidth}
      \begin{tikzpicture}[y=-1cm,scale=.75]
        \trianglestyles
        \begin{scope}[yshift=0cm]
          \fill[lightgray] (0,0) rectangle(7,1);
          \filldraw[draw=black,vertcolor] (0.5,0) rectangle ++ (1,1);
          \filldraw[draw=black,vertcolor] (1.5,0) rectangle ++ (1,1);
          \filldraw[draw=black,vertcolor] (2.5,0) rectangle ++ (1,1);
          \filldraw[draw=black,vertcolor] (3.5,0) rectangle ++ (1,1);
          \filldraw[draw=black,vertcolor] (4.5,0) rectangle ++ (1,1);
          \filldraw[draw=black,vertcolor] (5.5,0) rectangle ++ (1,1);
          \node[at={(1,.5)}, ptlabel] {$i_6$};
          \node[at={(2,.5)}, ptlabel] {$i_9$};
          \node[at={(3,.5)}, ptlabel] {$i_0$};
          \node[at={(4,.5)}, ptlabel] {$i_3$};
          \node[at={(5,.5)}, ptlabel] {$i_7$};
          \node[at={(6,.5)}, ptlabel] {$i_4$};
          \draw (0,0) -- (7,0);
          \draw (0,1) -- (7,1);
        \end{scope}

        \begin{scope}[yshift=-2cm]
          \begin{scope}[xshift=2cm]
            \filldraw[draw=black,vertcolor] (0,0) rectangle ++ (1,1);
            \filldraw[draw=black,vertcolor] (1,0) rectangle ++ (1,1);
            \node[at={(0.5,.5)}, ptlabel] {$d_0$};
            \node[at={(1.5,.5)}, ptlabel] {$d_1$};

            \draw (.5,-1) -- (0,0);
            \draw (1.5,-1) -- (2,0);
          \end{scope}
        \end{scope}
      \end{tikzpicture}
    \end{column}
  \end{columns}
\end{frame}

\begin{frame}[fragile]{Starting simple: P1}
  \begin{columns}
    \begin{column}{.5\textwidth}
      \centering

      \begin{tikzpicture}
        \stylesegmentsone
        \drawtriangle
        \filldraw [vertdof] (v0) node [doftext] {0} circle [radius=7pt];
        \filldraw [vertdof] (v1) node [doftext] {1} circle [radius=7pt];
        \filldraw [vertdof] (v2) node [doftext] {2} circle [radius=7pt];

        \tkzDefBarycentricPoint(v1=1,v2=1) \tkzGetPoint{to}
        \tkzDefShiftPoint[to](.2,.2){to1}

        % \begin{scope}[overlay]
        %   \tkzDefShiftPoint[v0](1.2,-.8){label0}
        %   \node [font=\scriptsize,at={(label0)}] {Reference element};
        %   \draw [-{stealth},draw=red] (label0) -- (v0);
        % \end{scope}

        \begin{scope}[overlay,xshift=4cm,yshift=1cm]
          \drawsmalltriangle
          \filldraw [vertdof] (v0) node [doftext] {$v_3$} circle [radius=7pt];
          \filldraw [vertdof] (v1) node [doftext] {$v_5$} circle [radius=7pt];
          \filldraw [vertdof] (v2) node [doftext] {$v_6$} circle [radius=7pt];

          \tkzDefBarycentricPoint(v0=1,v2=1) \tkzGetPoint{from}
          \tkzDefShiftPoint[from](-.3,0){from1}

          \draw [-{stealth},densely dashed] (from1) to [bend right=35] (to1);
        \end{scope}

      \end{tikzpicture}

      \vspace{4em}

      \begin{tikzpicture}[y=-1cm,scale=.75]
        \trianglestyles
        \begin{scope}[yshift=0cm]
          \fill[lightgray] (0,0) rectangle(7,1);
          \filldraw[draw=black,vertcolor] (0.5,0) rectangle ++ (1,1);
          \filldraw[draw=black,vertcolor] (1.5,0) rectangle ++ (1,1);
          \filldraw[draw=black,vertcolor] (2.5,0) rectangle ++ (1,1);
          \filldraw[draw=black,vertcolor] (3.5,0) rectangle ++ (1,1);
          \filldraw[draw=black,vertcolor] (4.5,0) rectangle ++ (1,1);
          \filldraw[draw=black,vertcolor] (5.5,0) rectangle ++ (1,1);
          \node[at={(1,.5)}, ptlabel] {$v_3$};
          \node[at={(2,.5)}, ptlabel] {$v_4$};
          \node[at={(3,.5)}, ptlabel] {$v_5$};
          \node[at={(4,.5)}, ptlabel] {$v_6$};
          \node[at={(5,.5)}, ptlabel] {$v_7$};
          \node[at={(6,.5)}, ptlabel] {$v_8$};
          \draw (0,0) -- (7,0);
          \draw (0,1) -- (7,1);
        \end{scope}
      \end{tikzpicture}
    \end{column}

    \hfill

    \begin{column}{.5\textwidth}
      \vspace{4em}
      \begin{minted}[
        frame=lines,
        framesep=2mm,
        baselinestretch=1.2,
        bgcolor=lightgray, fontsize=\tiny,
        linenos
      ]{python}
root = (
  MultiAxis()
  .add_part(AxisPart(nverts))
)
      \end{minted}
    \end{column}
  \end{columns}
\end{frame}

\begin{frame}[fragile]{Adding shape: vector P1}
  \begin{columns}
    \begin{column}{.5\textwidth}
      \centering

      \begin{tikzpicture}
        \stylesegmentsone
        \drawtriangle

        \filldraw [vertdof] (v0) ++ (.2,0) node [doftext] {1} circle [radius=7pt];
        \filldraw [vertdof] (v0) ++ (-.2,.1) node [doftext] {0} circle [radius=7pt];

        \filldraw [vertdof] (v1) ++ (.2,0) node [doftext] {3} circle [radius=7pt];
        \filldraw [vertdof] (v1) ++ (-.2,.1) node [doftext] {2} circle [radius=7pt];

        \filldraw [vertdof] (v2) ++ (.2,0) node [doftext] {5} circle [radius=7pt];
        \filldraw [vertdof] (v2) ++ (-.2,.1) node [doftext] {4} circle [radius=7pt];
      \end{tikzpicture}

      \vspace{2em}

      \begin{tikzpicture}[y=-1cm,scale=.75]
        \trianglestyles
        \begin{scope}[yshift=0cm]
          \fill[lightgray] (0,0) rectangle(7,1);
          \filldraw[draw=black,vertcolor] (0.5,0) rectangle ++ (1,1);
          \filldraw[draw=black,vertcolor] (1.5,0) rectangle ++ (1,1);
          \filldraw[draw=black,vertcolor] (2.5,0) rectangle ++ (1,1);
          \filldraw[draw=black,vertcolor] (3.5,0) rectangle ++ (1,1);
          \filldraw[draw=black,vertcolor] (4.5,0) rectangle ++ (1,1);
          \filldraw[draw=black,vertcolor] (5.5,0) rectangle ++ (1,1);
          \node[at={(1,.5)}, ptlabel] {$v_3$};
          \node[at={(2,.5)}, ptlabel] {$v_4$};
          \node[at={(3,.5)}, ptlabel] {$v_5$};
          \node[at={(4,.5)}, ptlabel] {$v_6$};
          \node[at={(5,.5)}, ptlabel] {$v_7$};
          \node[at={(6,.5)}, ptlabel] {$v_8$};
          \draw (0,0) -- (7,0);
          \draw (0,1) -- (7,1);
        \end{scope}

        \begin{scope}[yshift=-2cm]
          \begin{scope}[xshift=2cm]
            \filldraw[draw=black,vertcolor] (0,0) rectangle ++ (1,1);
            \filldraw[draw=black,vertcolor] (1,0) rectangle ++ (1,1);
            \node[at={(0.5,.5)}, ptlabel] {$d_0$};
            \node[at={(1.5,.5)}, ptlabel] {$d_1$};

            \draw (.5,-1) -- (0,0);
            \draw (1.5,-1) -- (2,0);
          \end{scope}
        \end{scope}

      \end{tikzpicture}
    \end{column}

    \hfill

    \begin{column}{.5\textwidth}
      \begin{minted}[
        frame=lines,
        framesep=2mm,
        baselinestretch=1.2,
        bgcolor=lightgray, fontsize=\tiny,
        linenos
      ]{python}
root = (
  MultiAxis()
  .add_part(AxisPart(nverts))
  .add_subaxis(AxisPart(2))
)
      \end{minted}
    \end{column}
  \end{columns}
\end{frame}

\begin{frame}[fragile]{Multiple entities: P2}
  \begin{columns}
    \begin{column}{.5\textwidth}
      \centering

      \begin{tikzpicture}
        \stylesegmentstwo
        \drawtriangle

        \filldraw [vertdof] (v0) node [doftext] {0} circle [radius=7pt];
        \filldraw [vertdof] (v1) node [doftext] {1} circle [radius=7pt];
        \filldraw [vertdof] (v2) node [doftext] {2} circle [radius=7pt];

        % edge dofs
        \tkzDefBarycentricPoint(v0=1,v1=1) \tkzGetPoint{e0d0}
        \filldraw [edgedof] (e0d0) node [doftext] {5} circle [radius=7pt];

        \tkzDefBarycentricPoint(v1=1,v2=1) \tkzGetPoint{e1d0}
        \filldraw [edgedof] (e1d0) node [doftext] {3} circle [radius=7pt];

        \tkzDefBarycentricPoint(v0=1,v2=1) \tkzGetPoint{e2d0}
        \filldraw [edgedof] (e2d0) node [doftext] {4} circle [radius=7pt];
      \end{tikzpicture}

      \vspace{2em}

      \begin{tikzpicture}[y=-1cm,scale=.75]
        \trianglestyles
        \begin{scope}[yshift=0cm]
          \fill[lightgray] (0,0) rectangle(7,1);
          \filldraw[draw=black,edgecolor] (0,0) rectangle ++ (1,1);
          \filldraw[draw=black,edgecolor] (1,0) rectangle ++ (1.5,1);
          \filldraw[draw=black,edgecolor] (2.5,0) rectangle ++ (1,1);
          \filldraw[draw=black,vertcolor] (3.5,0) rectangle ++ (1,1);
          \filldraw[draw=black,vertcolor] (4.5,0) rectangle ++ (1.5,1);
          \filldraw[draw=black,vertcolor] (6,0) rectangle ++ (1,1);
          \node[at={(.5,.5)}, ptlabel] {$e_0$};
          \node[at={(1.75,.5)}, ptlabel] {\dots};
          \node[at={(3,.5)}, ptlabel] {$e_m$};
          \node[at={(4,.5)}, ptlabel] {$v_0$};
          \node[at={(5.25,.5)}, ptlabel] {\dots};
          \node[at={(6.5,.5)}, ptlabel] {$v_n$};
          \draw (0,0) -- (7,0);
          \draw (0,1) -- (7,1);
        \end{scope}
      \end{tikzpicture}
    \end{column}

    \hfill

    \begin{column}{.5\textwidth}
      \checked p-adaptivity\footnotemark \\
      \checked Mixed meshes\footnotemark[\value{footnote}]

      \vspace{2em}

      \begin{minted}[
        frame=lines,
        framesep=2mm,
        baselinestretch=1.2,
        bgcolor=lightgray, fontsize=\tiny,
        linenos
      ]{python}
root = (
  MultiAxis()
  .add_part(AxisPart(nedges))
  .add_part(AxisPart(nverts))
)
      \end{minted}
    \end{column}
  \end{columns}
      \footnotetext{Since topological entities are now distinguishable}

\end{frame}

\begin{frame}[fragile]{Now with renumbering}
  \begin{columns}
    \begin{column}{.5\textwidth}
      \centering

      \begin{tikzpicture}
        \stylesegmentsone
        \drawtriangle

        \filldraw [vertdof] (v0) node [doftext] {0} circle [radius=7pt];
        \filldraw [vertdof] (v1) node [doftext] {1} circle [radius=7pt];
        \filldraw [vertdof] (v2) node [doftext] {2} circle [radius=7pt];

        % edge dofs
        \tkzDefBarycentricPoint(v0=1,v1=1) \tkzGetPoint{e0d0}
        \filldraw [edgedof] (e0d0) node [doftext] {5} circle [radius=7pt];

        \tkzDefBarycentricPoint(v1=1,v2=1) \tkzGetPoint{e1d0}
        \filldraw [edgedof] (e1d0) node [doftext] {3} circle [radius=7pt];

        \tkzDefBarycentricPoint(v0=1,v2=1) \tkzGetPoint{e2d0}
        \filldraw [edgedof] (e2d0) node [doftext] {4} circle [radius=7pt];
      \end{tikzpicture}

      \vspace{2em}

      \begin{tikzpicture}[y=-1cm,scale=.75]
        \trianglestyles
        \begin{scope}[yshift=0cm]
          \fill[lightgray] (0,0) rectangle(7,1);
          \filldraw[draw=black,edgecolor] (0.5,0) rectangle ++ (1,1);
          \filldraw[draw=black,vertcolor] (1.5,0) rectangle ++ (1,1);
          \filldraw[draw=black,edgecolor] (2.5,0) rectangle ++ (1,1);
          \filldraw[draw=black,edgecolor] (3.5,0) rectangle ++ (1,1);
          \filldraw[draw=black,vertcolor] (4.5,0) rectangle ++ (1,1);
          \filldraw[draw=black,vertcolor] (5.5,0) rectangle ++ (1,1);
          \node[at={(1,.5)}, ptlabel] {$e_2$};
          \node[at={(2,.5)}, ptlabel] {$v_7$};
          \node[at={(3,.5)}, ptlabel] {$e_1$};
          \node[at={(4,.5)}, ptlabel] {$e_0$};
          \node[at={(5,.5)}, ptlabel] {$v_4$};
          \node[at={(6,.5)}, ptlabel] {$v_2$};
          \draw (0,0) -- (7,0);
          \draw (0,1) -- (7,1);
        \end{scope}
      \end{tikzpicture}
    \end{column}

    \hfill

    \begin{column}{.5\textwidth}
      \checked Prescribe DoF ordering

      \vspace{2em}

      \begin{minted}[
        frame=lines,
        framesep=2mm,
        baselinestretch=1.2,
        bgcolor=lightgray, fontsize=\tiny,
        linenos
      ]{python}
root = (
  MultiAxis()
  .add_part(AxisPart(
    nedges,
    numbering=[4,2,5,...],
  ))
  .add_part(AxisPart(
    nverts,
    numbering=[3,0,1,...],
  ))
)
      \end{minted}
    \end{column}
  \end{columns}
\end{frame}

\begin{frame}[fragile]{More complicated inner shape: P3}
  \begin{columns}
    \begin{column}{.5\textwidth}
      \centering

      \begin{tikzpicture}
        \stylesegmentsone
        \drawtriangle

        \filldraw [vertdof] (v0) node [doftext] {0} circle [radius=7pt];
        \filldraw [vertdof] (v1) node [doftext] {1} circle [radius=7pt];
        \filldraw [vertdof] (v2) node [doftext] {2} circle [radius=7pt];

        % edge dofs
        \tkzDefBarycentricPoint(v0=2.3,v1=1) \tkzGetPoint{e0d0}
        \filldraw [edgedof] (e0d0) node [doftext] {7} circle [radius=7pt];

        \tkzDefBarycentricPoint(v0=1,v1=2.3) \tkzGetPoint{e0d1}
        \filldraw [edgedof] (e0d1) node [doftext] {8} circle [radius=7pt];

        \tkzDefBarycentricPoint(v1=2.3,v2=1) \tkzGetPoint{e1d0}
        \filldraw [edgedof] (e1d0) node [doftext] {3} circle [radius=7pt];

        \tkzDefBarycentricPoint(v1=1,v2=2.3) \tkzGetPoint{e1d1}
        \filldraw [edgedof] (e1d1) node [doftext] {4} circle [radius=7pt];

        \tkzDefBarycentricPoint(v0=2.3,v2=1) \tkzGetPoint{e2d0}
        \filldraw [edgedof] (e2d0) node [doftext] {5} circle [radius=7pt];

        \tkzDefBarycentricPoint(v0=1,v2=2.3) \tkzGetPoint{e2d1}
        \filldraw [edgedof] (e2d1) node [doftext] {6} circle [radius=7pt];

        % cell dof
        \tkzDefBarycentricPoint(v0=1,v1=1,v2=1) \tkzGetPoint{c0d0}
        \filldraw [celldof] (c0d0) node [doftext] {9} circle [radius=7pt];
      \end{tikzpicture}

      \vspace{2em}

      \begin{tikzpicture}[y=-1cm,scale=.75]
        \trianglestyles
        \begin{scope}[yshift=0cm]
          \fill[lightgray] (0,0) rectangle(7,1);
          \filldraw[draw=black,vertcolor] (0.5,0) rectangle ++ (1,1);
          \filldraw[draw=black,cellcolor] (1.5,0) rectangle ++ (1,1);
          \filldraw[draw=black,vertcolor] (2.5,0) rectangle ++ (1,1);
          \filldraw[draw=black,edgecolor] (3.5,0) rectangle ++ (1,1);
          \filldraw[draw=black,edgecolor] (4.5,0) rectangle ++ (1,1);
          \filldraw[draw=black,vertcolor] (5.5,0) rectangle ++ (1,1);
          \node[at={(1,.5)}, ptlabel] {$v_5$};
          \node[at={(2,.5)}, ptlabel] {$c_1$};
          \node[at={(3,.5)}, ptlabel] {$v_4$};
          \node[at={(4,.5)}, ptlabel] {$e_4$};
          \node[at={(5,.5)}, ptlabel] {$e_1$};
          \node[at={(6,.5)}, ptlabel] {$v_2$};
          \draw (0,0) -- (7,0);
          \draw (0,1) -- (7,1);
        \end{scope}

        \begin{scope}[yshift=-2cm]
          \begin{scope}[xshift=1.5cm]
            \filldraw[draw=black,cellcolor] (0,0) rectangle ++ (1,1);
            \node[at={(0.5,.5)}, ptlabel] {$d_0$};

            \draw (0,-1) -- (0,0);
            \draw (1,-1) -- (1,0);
          \end{scope}

          \begin{scope}[xshift=3cm]
            \filldraw[draw=black,edgecolor] (0,0) rectangle ++ (1,1);
            \filldraw[draw=black,edgecolor] (1,0) rectangle ++ (1,1);
            \node[at={(0.5,.5)}, ptlabel] {$d_0$};
            \node[at={(1.5,.5)}, ptlabel] {$d_1$};

            \draw (.5,-1) -- (0,0);
            \draw (1.5,-1) -- (2,0);
          \end{scope}
          \begin{scope}[xshift=5.5cm]
            \filldraw[draw=black,vertcolor] (0,0) rectangle ++ (1,1);
            \node[at={(0.5,.5)}, ptlabel] {$d_0$};

            \draw (0,-1) -- (0,0);
            \draw (1,-1) -- (1,0);
          \end{scope}
        \end{scope}

      \end{tikzpicture}
    \end{column}

    \hfill

    \begin{column}{.5\textwidth}
      \begin{minted}[
        frame=lines,
        framesep=2mm,
        baselinestretch=1.2,
        bgcolor=lightgray, fontsize=\tiny,
        linenos
      ]{python}
root = (
  MultiAxis()
  .add_part(AxisPart(ncells, "cells"))
  .add_part(AxisPart(nedges, "edges"))
  .add_part(AxisPart(nverts, "verts"))
  .add_subaxis("edges", AxisPart(2))
)
      \end{minted}
    \end{column}
  \end{columns}
\end{frame}

\begin{frame}[fragile]{Mixed}
      \centering

      \begin{tikzpicture}[y=-1cm,scale=.75]
        \trianglestyles
        \mixedstylesetup
        \begin{scope}[xshift=3.25cm, yshift=0cm]
          \filldraw[fill=white,draw=black] (0,0) rectangle (1,1);
          \filldraw[fill=white,draw=black] (1,0) rectangle (2,1);
          \node[at={(.5,.5)}, ptlabel] {$V_0$};
          \node[at={(1.5,.5)}, ptlabel] {$V_1$};
        \end{scope}

        \begin{scope}[yshift=-2cm]
          \begin{scope}[xshift=0cm]
            \fill[lightgray] (0,0) rectangle (4,1);
            \filldraw[draw=black,cellcolor] (0.5,0) rectangle (1.5,1);
            \filldraw[draw=black,vertcolor] (1.5,0) rectangle (2.5,1);
            \filldraw[draw=black,edgecolor] (2.5,0) rectangle (3.5,1);
            \node[at={(1,.5)},ptlabel] {$c_0$};
            \node[at={(2,.5)},ptlabel] {$v_1$};
            \node[at={(3,.5)},ptlabel] {$e_1$};
            \draw (0,0) -- (4,0);
            \draw (0,1) -- (4,1);
          \end{scope}

          \begin{scope}[xshift=4.5cm]
            \fill[lightgray] (0,0) rectangle (4,1);
            \filldraw[draw=black,cellcolor] (0.5,0) rectangle (1.5,1);
            \filldraw[draw=black,vertcolor] (1.5,0) rectangle (2.5,1);
            \filldraw[draw=black,edgecolor] (2.5,0) rectangle (3.5,1);
            \node[at={(1,.5)},ptlabel] {$c_0$};
            \node[at={(2,.5)},ptlabel] {$v_1$};
            \node[at={(3,.5)},ptlabel] {$e_1$};
            \draw (0,0) -- (4,0);
            \draw (0,1) -- (4,1);
          \end{scope}
        \end{scope}

        \draw [densely dashed] (3.25,1) -- (0,2);
        \draw [densely dashed] (4.25,1) -- (4,2);
        \draw [densely dashed] (4.25,1) -- ({0+4.5},2);
        \draw [densely dashed] (5.25,1) -- ({4+4.5},2);
      \end{tikzpicture}

      \centering
      \begin{minipage}{.6\textwidth}
      \begin{minted}[
        frame=lines,
        framesep=2mm,
        baselinestretch=1.2,
        bgcolor=lightgray, fontsize=\tiny,
        linenos
      ]{python}
root = (
  MultiAxis()
  .add_part(AxisPart(1, "V0"))
  .add_part(AxisPart(1, "V1"))
  .add_subaxis("V0", ...)
  .add_subaxis("V1", ...)
)
      \end{minted}
    \end{minipage}
\end{frame}

\begin{frame}{\pyop3 wishlist}
  \textbf{Features:}

  \unchecked Handle orientations \\
  \checked p-adaptivity \\
  \checked Mixed meshes \\

  \textbf{Performance:}

  \unchecked Exploit mesh partial structure \\
  \checked Prescribe DoF ordering
\end{frame}

\begin{frame}{Partially-structured meshes: extruded}
  \begin{tikzpicture}[scale=.5]
    \tkzDefPoint(0,0){v0v0}
    \tkzDefShiftPoint[v0v0](2,0){v1v0}
    \tkzDefShiftPoint[v1v0](2,0){v2v0}
    \tkzDefShiftPoint[v0v0](0,2){v0v1}
    \tkzDefShiftPoint[v1v0](0,2){v1v1}
    \tkzDefShiftPoint[v2v0](0,2){v2v1}
    \tkzDefShiftPoint[v0v1](0,2){v0v2}
    \tkzDefShiftPoint[v1v1](0,2){v1v2}
    \tkzDefShiftPoint[v2v1](0,2){v2v2}
    \tkzDefShiftPoint[v0v2](0,2){v0v3}
    \tkzDefShiftPoint[v1v2](0,2){v1v3}

    % horiz edges
    \tkzDrawSegments(v0v0,v1v0 v1v0,v2v0)
    \tkzDrawSegments(v0v1,v1v1 v1v1,v2v1)
    \tkzDrawSegments(v0v2,v1v2 v1v2,v2v2)
    \tkzDrawSegments(v0v3,v1v3)

    % vert edges
    \tkzDrawSegments(v0v0,v0v1 v0v1,v0v2 v0v2,v0v3)
    \tkzDrawSegments(v1v0,v1v1 v1v1,v1v2 v1v2,v1v3)
    \tkzDrawSegments(v2v0,v2v1 v2v1,v2v2)

    % verts
    \tkzDrawPoints(v0v0,v0v1,v0v2,v0v3)
    \tkzDrawPoints(v1v0,v1v1,v1v2,v1v3)
    \tkzDrawPoints(v2v0,v2v1,v2v2)

    % add labels

    % lhs
    \tkzDefBarycentricPoint(v0v0=1,v0v1=1) \tkzGetPoint{v0e0}
    \tkzDefBarycentricPoint(v0v1=1,v0v2=1) \tkzGetPoint{v0e1}
    \tkzDefBarycentricPoint(v0v2=1,v0v3=1) \tkzGetPoint{v0e2}

    \node [xshift=-1.2cm] (v0v0label) at (v0v0) {$(v_0,v_0)$};
    \node [xshift=-1.2cm] (v0v1label) at (v0v1) {$(v_0,v_1)$};
    \node [xshift=-1.2cm] (v0v2label) at (v0v2) {$(v_0,v_2)$};
    \node [xshift=-1.2cm] (v0v3label) at (v0v3) {$(v_0,v_3)$};

    \node [xshift=-1.2cm] (v0e0label) at (v0e0) {$(v_0,e_0)$};
    \node [xshift=-1.2cm] (v0e1label) at (v0e1) {$(v_0,e_1)$};
    \node [xshift=-1.2cm] (v0e2label) at (v0e2) {$(v_0,e_2)$};

    \draw [-{stealth},shorten >=2pt] (v0v0label) -- (v0v0);
    \draw [-{stealth},shorten >=2pt] (v0v1label) -- (v0v1);
    \draw [-{stealth},shorten >=2pt] (v0v2label) -- (v0v2);
    \draw [-{stealth},shorten >=2pt] (v0v3label) -- (v0v3);

    \draw [-{stealth},shorten >=2pt] (v0e0label) -- (v0e0);
    \draw [-{stealth},shorten >=2pt] (v0e1label) -- (v0e1);
    \draw [-{stealth},shorten >=2pt] (v0e2label) -- (v0e2);

    % rhs
    \tkzDefBarycentricPoint(v1v0=1,v2v0=1) \tkzGetPoint{e1e0}
    \tkzDefBarycentricPoint(v1v1=1,v2v1=1) \tkzGetPoint{e1e1}
    \tkzDefBarycentricPoint(v1v2=1,v2v2=1) \tkzGetPoint{e1e2}

    \tkzDefBarycentricPoint(v1v0=1,v1v1=1,v2v0=1,v2v1=1) \tkzGetPoint{e1c0}
    \tkzDefBarycentricPoint(v1v1=1,v1v2=1,v2v1=1,v2v2=1) \tkzGetPoint{e1c1}

    \node [xshift=1.9cm,yshift=.5cm] (e1e0label) at (e1e0) {$(e_1,e_0)$};
    \node [xshift=1.9cm,yshift=.5cm] (e1e1label) at (e1e1) {$(e_1,e_1)$};
    \node [xshift=1.9cm,yshift=.5cm] (e1e2label) at (e1e2) {$(e_1,e_2)$};

    \node [xshift=1.9cm,yshift=.5cm] (e1c0label) at (e1c0) {$(e_1,c_0)$};
    \node [xshift=1.9cm,yshift=.5cm] (e1c1label) at (e1c1) {$(e_1,c_1)$};

    \draw [-{stealth},shorten >=2pt] (e1e0label.west) -- (e1e0);
    \draw [-{stealth},shorten >=2pt] (e1e1label.west) -- (e1e1);
    \draw [-{stealth},shorten >=2pt] (e1e2label.west) -- (e1e2);

    \draw [-{stealth},shorten >=2pt] (e1c0label.west) -- (e1c0);
    \draw [-{stealth},shorten >=2pt] (e1c1label.west) -- (e1c1);

    % bottom labels need to be included in bounding box
    \tkzDefBarycentricPoint(v0v0=1,v1v0=1) \tkzGetPoint{e0}
    \tkzLabelPoint[below](e0){$e_5$}
    \tkzDefBarycentricPoint(v1v0=1,v2v0=1) \tkzGetPoint{e1}
    \tkzLabelPoint[below](e1){$e_1$}

    \tkzLabelPoint[below](v0v0){$v_0$}
    \tkzLabelPoint[below](v1v0){$v_3$}
    \tkzLabelPoint[below](v2v0){$v_8$}
  \end{tikzpicture}

  \centering

  \quad
  \vspace{-2em}

  % and data layout
  \begin{tikzpicture}[y=-1cm,scale=.75]
    \trianglestyles
    \begin{scope}[xshift=3.25cm, yshift=0cm]
      \filldraw[draw=black,fill=white] (0,0) rectangle ++ (1,1);
      \filldraw[draw=black,fill=white] (1,0) rectangle ++ (1,1);
      \filldraw[draw=black,fill=white] (2,0) rectangle ++ (1,1);
      \filldraw[draw=black,fill=white] (3,0) rectangle ++ (1,1);
      \filldraw[draw=black,fill=white] (4,0) rectangle ++ (1,1);

      \node[at={(.5,.5)}, ptlabel] {$v_0$};
      \node[at={(1.5,.5)}, ptlabel] {$v_8$};
      \node[at={(2.5,.5)}, ptlabel] {$e_5$};
      \node[at={(3.5,.5)}, ptlabel] {$v_3$};
      \node[at={(4.5,.5)}, ptlabel] {$e_1$};
    \end{scope}

    \begin{scope}[yshift=-2cm]
      \begin{scope}[xshift=1cm]
        \fill[lightgray] (0,0) rectangle (3.5,1);
        \filldraw[draw=black,vertcolor] (0,0) rectangle ++ (1,1);
        \filldraw[draw=black,edgecolor] (1,0) rectangle ++ (1,1);
        \filldraw[draw=black,vertcolor] (2,0) rectangle ++ (1,1);
        \node[at={(.5,.5)}, ptlabel] {$v_0$};
        \node[at={(1.5,.5)}, ptlabel] {$e_0$};
        \node[at={(2.5,.5)}, ptlabel] {$v_1$};
        \draw (0,0) -- (3.5,0);
        \draw (0,1) -- (3.5,1);
      \end{scope}

      \begin{scope}[xshift=6cm]
        \fill[lightgray] (0,0) rectangle (3.5,1);
        \filldraw[draw=black,edgecolor] (0,0) rectangle ++ (1,1);
        \filldraw[draw=black,cellcolor] (1,0) rectangle ++ (1,1);
        \filldraw[draw=black,edgecolor] (2,0) rectangle ++ (1,1);
        \node[at={(.5,.5)}, ptlabel] {$e_0$};
        \node[at={(1.5,.5)}, ptlabel] {$c_0$};
        \node[at={(2.5,.5)}, ptlabel] {$e_1$};
        \draw (0,0) -- (3.5,0);
        \draw (0,1) -- (3.5,1);
      \end{scope}
    \end{scope}

    \draw ({3.25+0},1) -- ({0+1},2);
    \draw [densely dashed] ({3.25+1},1) -- ({0+4.5},2);
    \draw ({3.25+4},1) -- ({6+0},2);
    \draw [densely dashed] ({3.25+5},1) -- ({6+3.5},2);
  \end{tikzpicture}

  This might be overkill... but it works!
\end{frame}
\begin{frame}{Orientation: P3}
  \centering

  \begin{tikzpicture}
    \stylesegmentsone
    \drawtriangle

      \filldraw [vertdof] (v0) node [doftext] {0} circle [radius=7pt];
      \filldraw [vertdof] (v1) node [doftext] {1} circle [radius=7pt];
      \filldraw [vertdof] (v2) node [doftext] {2} circle [radius=7pt];

      % edge dofs
      \tkzDefBarycentricPoint(v0=2.3,v1=1) \tkzGetPoint{e0d0}
      \filldraw [edgedof] (e0d0) node [doftext] {7} circle [radius=7pt];

      \tkzDefBarycentricPoint(v0=1,v1=2.3) \tkzGetPoint{e0d1}
      \filldraw [edgedof] (e0d1) node [doftext] {8} circle [radius=7pt];

      \tkzDefBarycentricPoint(v1=2.3,v2=1) \tkzGetPoint{e1d0}
      \filldraw [edgedof] (e1d0) node [doftext] {3} circle [radius=7pt];

      \tkzDefBarycentricPoint(v1=1,v2=2.3) \tkzGetPoint{e1d1}
      \filldraw [edgedof] (e1d1) node [doftext] {4} circle [radius=7pt];

      \tkzDefBarycentricPoint(v0=2.3,v2=1) \tkzGetPoint{e2d0}
      \filldraw [edgedof] (e2d0) node [doftext] {5} circle [radius=7pt];

      \tkzDefBarycentricPoint(v0=1,v2=2.3) \tkzGetPoint{e2d1}
      \filldraw [edgedof] (e2d1) node [doftext] {6} circle [radius=7pt];

      % cell dof
      \tkzDefBarycentricPoint(v0=1,v1=1,v2=1) \tkzGetPoint{c0d0}
      \filldraw [celldof] (c0d0) node [doftext] {9} circle [radius=7pt];

      % flipped arrow
      \tkzDefShiftPoint[e1d0](0,.3){e1d0l}
      \tkzDefShiftPoint[e1d1](.3,0){e1d1l}
      \draw [{stealth}-{stealth},draw=red,line width=1pt] (e1d0l) to [bend right=40] (e1d1l);

      \node at (3.5,1) {vs};

      % now flipped
      \begin{scope}[xshift=5cm]
        \drawflippedtriangle

          \filldraw [vertdof] (v0) node [doftext] {0} circle [radius=7pt];
          \filldraw [vertdof] (v1) node [doftext] {1} circle [radius=7pt];
          \filldraw [vertdof] (v2) node [doftext] {2} circle [radius=7pt];

          % edge dofs
          \tkzDefBarycentricPoint(v0=2.3,v1=1) \tkzGetPoint{e0d0}
          \filldraw [edgedof] (e0d0) node [doftext] {7} circle [radius=7pt];

          \tkzDefBarycentricPoint(v0=1,v1=2.3) \tkzGetPoint{e0d1}
          \filldraw [edgedof] (e0d1) node [doftext] {8} circle [radius=7pt];

          \tkzDefBarycentricPoint(v1=2.3,v2=1) \tkzGetPoint{e1d0}
          \filldraw [edgedof] (e1d0) node [doftext] {4} circle [radius=7pt];

          \tkzDefBarycentricPoint(v1=1,v2=2.3) \tkzGetPoint{e1d1}
          \filldraw [edgedof] (e1d1) node [doftext] {3} circle [radius=7pt];

          \tkzDefBarycentricPoint(v0=2.3,v2=1) \tkzGetPoint{e2d0}
          \filldraw [edgedof] (e2d0) node [doftext] {5} circle [radius=7pt];

          \tkzDefBarycentricPoint(v0=1,v2=2.3) \tkzGetPoint{e2d1}
          \filldraw [edgedof] (e2d1) node [doftext] {6} circle [radius=7pt];

          % cell dof
          \tkzDefBarycentricPoint(v0=1,v1=1,v2=1) \tkzGetPoint{c0d0}
          \filldraw [celldof] (c0d0) node [doftext] {9} circle [radius=7pt];
        \end{scope}
      \end{tikzpicture}

      \vspace{2em}

      \centering
      \begin{tikzpicture}[y=-1cm,scale=.75]
        \trianglestyles
        \begin{scope}[yshift=0cm]
          \fill[lightgray] (0,0) rectangle(7,1);
          \filldraw[draw=black,vertcolor] (0.5,0) rectangle ++ (1,1);
          \filldraw[draw=black,cellcolor] (1.5,0) rectangle ++ (1,1);
          \filldraw[draw=black,vertcolor] (2.5,0) rectangle ++ (1,1);
          \filldraw[draw=black,edgecolor] (3.5,0) rectangle ++ (1,1);
          \filldraw[draw=black,edgecolor] (4.5,0) rectangle ++ (1,1);
          \filldraw[draw=black,vertcolor] (5.5,0) rectangle ++ (1,1);
          \node[at={(1,.5)}, ptlabel] {$v_5$};
          \node[at={(2,.5)}, ptlabel] {$c_1$};
          \node[at={(3,.5)}, ptlabel] {$v_4$};
          \node[at={(4,.5)}, ptlabel] {$e_4$};
          \node[at={(5,.5)}, ptlabel] {$e_1$};
          \node[at={(6,.5)}, ptlabel] {$v_2$};
          \draw (0,0) -- (7,0);
          \draw (0,1) -- (7,1);
        \end{scope}

        \begin{scope}[yshift=-2cm]
          \begin{scope}[xshift=1.5cm]
            \filldraw[draw=black,cellcolor] (0,0) rectangle ++ (1,1);
            \node[at={(0.5,.5)}, ptlabel] {$d_0$};

            \draw (0,-1) -- (0,0);
            \draw (1,-1) -- (1,0);
          \end{scope}

          \begin{scope}[xshift=3cm]
            \filldraw[draw=black,edgecolor] (0,0) rectangle ++ (1,1);
            \filldraw[draw=black,edgecolor] (1,0) rectangle ++ (1,1);
            \node[at={(0.5,.5)}, ptlabel] {$d_0$};
            \node[at={(1.5,.5)}, ptlabel] {$d_1$};

            \draw (.5,-1) -- (0,0);
            \draw (1.5,-1) -- (2,0);

            % flipped arrow
            \draw [{stealth}-{stealth},draw=red,line width=1pt] (.5,1.1) to [bend right=70] (1.5,1.1);
          \end{scope}
          \begin{scope}[xshift=5.5cm]
            \filldraw[draw=black,vertcolor] (0,0) rectangle ++ (1,1);
            \node[at={(0.5,.5)}, ptlabel] {$d_0$};

            \draw (0,-1) -- (0,0);
            \draw (1,-1) -- (1,0);
          \end{scope}
        \end{scope}

      \end{tikzpicture}
    \end{frame}

\begin{frame}{Orientation: Raviart-Thomas}
  \centering

  \begin{tikzpicture}
    \stylesegmentstwo
    \drawtriangle

    \tkzDefBarycentricPoint(v0=1,v1=1) \tkzGetPoint{e0d0}
    \tkzDefShiftPoint[e0d0](-90:.8){e0d0v}
    \draw [vdof] (e0d0) -- (e0d0v);
    \filldraw [edgedof] (e0d0) node [doftext] {2} circle [radius=7pt];

    \tkzDefBarycentricPoint(v0=1,v2=1) \tkzGetPoint{e1d0}
    \tkzDefShiftPoint[e1d0](180:.8){e1d0v}
    \draw [vdof] (e1d0) -- (e1d0v);
    \filldraw [edgedof] (e1d0) node [doftext] {1} circle [radius=7pt];

    \tkzDefBarycentricPoint(v1=1,v2=1) \tkzGetPoint{e2d0}
    \tkzDefShiftPoint[e2d0](225:.8){e2d0v}
    \draw [vdof] (e2d0) -- (e2d0v);
    \filldraw [edgedof] (e2d0) node [doftext] {0} circle [radius=7pt];

    \node at (3.5,1) {vs};

    % now flipped
    \begin{scope}[xshift=5cm]
      \drawflippedtriangle
      \tkzDefBarycentricPoint(v0=1,v1=1) \tkzGetPoint{e0d0}
      \tkzDefShiftPoint[e0d0](-90:.8){e0d0v}
      \draw [vdof] (e0d0) -- (e0d0v);
      \filldraw [edgedof] (e0d0) node [doftext] {2} circle [radius=7pt];

      \tkzDefBarycentricPoint(v0=1,v2=1) \tkzGetPoint{e1d0}
      \tkzDefShiftPoint[e1d0](180:.8){e1d0v}
      \draw [vdof] (e1d0) -- (e1d0v);
      \filldraw [edgedof] (e1d0) node [doftext] {1} circle [radius=7pt];

      \tkzDefBarycentricPoint(v1=1,v2=1) \tkzGetPoint{e2d0}
      \tkzDefShiftPoint[e2d0](45:.8){e2d0v}
      \draw [vdof] (e2d0) -- (e2d0v);
      \filldraw [edgedof] (e2d0) node [doftext] {0} circle [radius=7pt];
    \end{scope}
  \end{tikzpicture}

  \vspace{2em}

  \begin{tikzpicture}[y=-1cm,scale=.75]
    \trianglestyles
    \fill[lightgray] (0,0) rectangle(7,1);
    \filldraw[draw=black,edgecolor] (0.5,0) rectangle ++ (1,1);
    \filldraw[draw=black,edgecolor] (1.5,0) rectangle ++ (1,1);
    \filldraw[draw=black,edgecolor] (2.5,0) rectangle ++ (1,1);
    \filldraw[draw=black,edgecolor] (3.5,0) rectangle ++ (1,1);
    \filldraw[draw=black,edgecolor] (4.5,0) rectangle ++ (1,1);
    \filldraw[draw=black,edgecolor] (5.5,0) rectangle ++ (1,1);
    \node[at={(1,.5)}, ptlabel] {$e_5$};
    \node[at={(2,.5)}, ptlabel] {$e_9$};
    \node[at={(3,.5)}, ptlabel] {$e_0$};
    \node[at={(4,.5)}, ptlabel] {$e_4$};
    \node[at={(5,.5)}, ptlabel] {$e_1$};
    \node[at={(6,.5)}, ptlabel] {$e_2$};
    \draw (0,0) -- (7,0);
    \draw (0,1) -- (7,1);

    % times minus 1
    \draw [-{stealth},draw=red,shorten <=5pt,line width=1pt] (2.8,2.4) node {$(\times -\!1)$} -- (3,1.1);
  \end{tikzpicture}
\end{frame}



\begin{frame}{\pyop3 wishlist}
  \textbf{Features:}

  \checked Handle orientations \\
  \checked p-adaptivity \\
  \checked Mixed meshes \\

  \textbf{Performance:}

  \checked Exploit mesh partial structure \\
  \checked Prescribe DoF ordering
\end{frame}

\begin{frame}{Summary}
  \begin{itemize}
    \item \pyop3 has a unifying abstraction for all* of the data structures currently used in \pyop2
    \item This abstraction should let us do a lot of cool things, automatically!
  \end{itemize}

  \vspace{2em}
  \small{*I claim}
\end{frame}

\begin{frame}{Things I missed}
  \small
  \begin{itemize}
    \item The cool new interface (inc. map and loop composition)
    \item Tight integration with PETSc (esp. DMPlex)
    \item Support for sparse matrices
    \item Support for ragged data structures (e.g. variable layer extrusion, PIC, mixed-arity maps)
    \item MPI parallelism
    \item Could streamline PCPATCH and multigrid code (via loop/map composition)
    \item Should retain \pyop2's work on GPUs and inter-element vectorisation
    \item Additional data layout transformations/optimisations
    \item Could potentially do a similar mesh structure trick for refined meshes
  \end{itemize}
\end{frame}

\section{Appendix}

\begin{frame}[fragile]{\pyop3 interface}
  \begin{minted}[fontsize=\scriptsize]{python}
do_loop(
  c := mesh.cells.index,
  kernel(dat0[closure(c)], dat1[closure(c)])
)
  \end{minted}

  \begin{minted}[fontsize=\scriptsize]{python}
do_loop(
  f := mesh.interior_facets.index,
  kernel(
    dat0[closure(support(f))],
    dat1[closure(support(f))]
  )
)
  \end{minted}
\end{frame}

\begin{frame}{DMPlex 1}
\begin{figure}
  \centering
  \begin{subfigure}{0.45\textwidth}
    \begin{tikzpicture}[scale=.5]
      \tkzDefPoint(0,2){v0}
      \tkzDefPoint(2,4){v1}
      \tkzDefPoint(2,0){v2}
      \tkzDefPoint(4,2){v3}

      \tkzDrawSegments(v0,v1 v0,v2 v1,v2 v1,v3 v2,v3)
      \tkzDrawPoints(v0,v1,v2,v3)

      \tkzDefBarycentricPoint(v0=1.2,v1=1,v2=1) \tkzGetPoint{c0}
      \tkzLabelPoint[centered](c0){1}
      \tkzDefBarycentricPoint(v1=1,v2=1,v3=1.2) \tkzGetPoint{c1}
      \tkzLabelPoint[centered](c1){2}

      \tkzLabelPoint[left](v0){3}
      \tkzLabelPoint[above](v1){4}
      \tkzLabelPoint[below](v2){5}
      \tkzLabelPoint[right](v3){6}

      \tkzLabelSegment[below left](v0,v2){7}
      \tkzLabelSegment[above left](v0,v1){8}
      \tkzLabelSegment[left](v1,v2){9}
      \tkzLabelSegment[above right](v1,v3){10}
      \tkzLabelSegment[below right](v2,v3){11}
    \end{tikzpicture}
  \end{subfigure}
  %
  \begin{subfigure}{0.45\textwidth}
    \centering
    \begin{tikzpicture}[scale=.5]
      \basichasse
    \end{tikzpicture}
  \end{subfigure}
  \caption{
    An example mesh and its Hasse diagram representation.
    Note that the topological entities are numbered according to the DMPlex convention of first cells, then vertices, then faces.
  }
  \label{fig:hasse_diagram}
\end{figure}
\end{frame}

\begin{frame}{DMPlex 2}
\begin{figure}
  \centering
  \begin{subfigure}{0.45\textwidth}
    \centering
    \begin{tikzpicture}[scale=.5]
      \basichasse
      \drawpolygon 7,9;
      \draw [dashed] (1) circle [radius=15pt];
    \end{tikzpicture}
    \caption{$\cone(1) = \{7,8,9\}$}
  \end{subfigure}
  %
  \begin{subfigure}{0.45\textwidth}
    \centering
    \begin{tikzpicture}[scale=.5]
      \basichasse
      \drawpolygon 1,2;
      \draw [dashed] (9) circle [radius=15pt];
    \end{tikzpicture}
    \caption{$\support(9) = \{1,2\}$}
  \end{subfigure}
  %
  \begin{subfigure}{0.45\textwidth}
    \centering
    \begin{tikzpicture}[scale=.5]
      \basichasse
      \drawpolygon 1,7,3,5,9;
    \end{tikzpicture}
    \caption{$\closure(1) = \{1,3,4,5,7,8,9\}$}
  \end{subfigure}
  %
  \begin{subfigure}{0.45\textwidth}
    \centering
    \begin{tikzpicture}[scale=.5]
      \basichasse
      \drawpolygon 4,10,2,1,8;
    \end{tikzpicture}
    \caption{$\plexstar(4) = \{4,8,9,10,1,2\}$}
  \end{subfigure}

  \caption{The possible DMPlex covering queries (applied to the Hasse diagram from Figure~\ref{fig:hasse_diagram}).}
  \label{fig:plex_restrictions}
\end{figure}
\end{frame}

\begin{frame}{Mixed reordering}
\begin{figure}
  \centering
  \begin{subfigure}{.65\textwidth}
    \centering
    \begin{tikzpicture}[y=-1cm,scale=.35]
      \mixedstylesetup
      \begin{scope}[xshift=3.25cm, yshift=0cm]
        \filldraw[v0,draw=black] (0,0) rectangle (1,1);
        \filldraw[v1,draw=black] (1,0) rectangle (2,1);
        \node[at={(.5,.5)}, ptlabel] {$V_0$};
        \node[at={(1.5,.5)}, ptlabel] {$V_1$};
      \end{scope}

      \begin{scope}[yshift=-2cm]
        \begin{scope}[xshift=0cm]
          \fill[lightgray] (0,0) rectangle (4,1);
          \filldraw[draw=black, fill=white] (0.5,0) rectangle (1.5,1);
          \filldraw[draw=black, fill=white] (1.5,0) rectangle (2.5,1);
          \filldraw[draw=black, fill=white] (2.5,0) rectangle (3.5,1);
          \node[at={(1,.5)}, ptlabel] {$c_0$};
          \node[at={(2,.5)}, ptlabel] {$v_1$};
          \node[at={(3,.5)}, ptlabel] {$c_4$};
          \draw (0,0) -- (4,0);
          \draw (0,1) -- (4,1);
        \end{scope}

        \begin{scope}[xshift=4.5cm]
          \fill[lightgray] (0,0) rectangle (4,1);
          \filldraw[draw=black, fill=white] (0.5,0) rectangle (1.5,1);
          \filldraw[draw=black, fill=white] (1.5,0) rectangle (2.5,1);
          \filldraw[draw=black, fill=white] (2.5,0) rectangle (3.5,1);
          \node[at={(1,.5)}, ptlabel] {$c_0$};
          \node[at={(2,.5)}, ptlabel] {$v_1$};
          \node[at={(3,.5)}, ptlabel] {$c_4$};
          \draw (0,0) -- (4,0);
          \draw (0,1) -- (4,1);
        \end{scope}
      \end{scope}

      \draw [densely dashed] (3.25,1) -- (0,2);
      \draw [densely dashed] (4.25,1) -- (4,2);
      \draw [densely dashed] (4.25,1) -- ({0+4.5},2);
      \draw [densely dashed] (5.25,1) -- ({4+4.5},2);

      \node [at={(9.5,.5)},anchor=center] {Spaces};
      \node [at={(9.5,2.5)},anchor=center] {Points};
    \end{tikzpicture}
    \caption{
      A typical data layout for a `mixed' system with the spaces $V_0$ and $V_1$ forming the `outer' axis.
    }
    \label{fig:mixedreorder_outer}
  \end{subfigure}
  \hfill
  \begin{subfigure}{.3\textwidth}
    \centering
    \begin{tikzpicture}[x=.8cm,y=-1cm,scale=.5]
      \mixedstylesetup
      \draw (0,0) .. controls (-.2,0) and (-.2,3) .. (0,3);
      \draw (1,0) .. controls (1.2,0) and (1.2,3) .. (1,3);
      \filldraw [v0,rounded corners,draw=none]
        (.1,.05) -- (.9,.05) -- (.9,1.45) -- (.1,1.45) -- cycle;
      \filldraw [v1,rounded corners,draw=none]
        (.1,1.55) -- (.9,1.55) -- (.9,2.95) -- (.1,2.95) -- cycle;
    \end{tikzpicture}
    \caption{The resulting block-structured vector.}
    \label{fig:mixedreorder_outer_vec}
  \end{subfigure}

  \vspace{1em}
 
  \begin{subfigure}{.65\textwidth}
    \centering
    \begin{tikzpicture}[y=-1cm,scale=.35]
      \mixedstylesetup
      \begin{scope}[xshift=0cm,yshift=0cm]
        \fill[lightgray] (0,0) rectangle (4,1);
        \filldraw[draw=black, fill=white] (0.5,0) rectangle (1.5,1);
        \filldraw[draw=black, fill=white] (1.5,0) rectangle (2.5,1);
        \filldraw[draw=black, fill=white] (2.5,0) rectangle (3.5,1);
        \node[at={(1,.5)}, ptlabel] {$c_0$};
        \node[at={(2,.5)}, ptlabel] {$v_1$};
        \node[at={(3,.5)}, ptlabel] {$c_4$};
        \draw (0,0) -- (4,0);
        \draw (0,1) -- (4,1);
      \end{scope}

      \begin{scope}[xshift=1cm, yshift=-2cm]
        \filldraw[v0,draw=black] (0,0) rectangle (1,1);
        \filldraw[v1,draw=black] (1,0) rectangle (2,1);
        \node[at={(.5,.5)}, ptlabel] {$V_0$};
        \node[at={(1.5,.5)}, ptlabel] {$V_1$};
      \end{scope}

      \draw (1.5,1) -- (1,2);
      \draw (2.5,1) -- (3,2);

      \node [at={(5,.5)},anchor=center] {Points};
      \node [at={(5,2.5)},anchor=center] {Spaces};
    \end{tikzpicture}
    \caption{A transformed data layout where the ``Spaces" and ``Points" axes have been swapped.}
    \label{fig:mixedreorder_inner}
  \end{subfigure}
  \hfill
  \begin{subfigure}{.3\textwidth}
    \centering
    \begin{tikzpicture}[x=.8cm,y=-1cm,scale=.5]
      \mixedstylesetup
      \tikzstyle{entry} = [rounded corners,draw=none];
      \tikzstyle{blue} = [v0,entry];
      \tikzstyle{red} = [v1,entry];
      \draw (0,0) .. controls (-.2,0) and (-.2,3) .. (0,3);
      \draw (1,0) .. controls (1.2,0) and (1.2,3) .. (1,3);
      \filldraw [blue] (.1,.05) -- (.9,.05) -- (.9,.45) -- (.1,.45) -- cycle;
      \filldraw [red] (.1,.55) -- (.9,.55) -- (.9,.95) -- (.1,.95) -- cycle;
      \filldraw [blue] (.1,1.05) -- (.9,1.05) -- (.9,1.45) -- (.1,1.45) -- cycle;
      \filldraw [red] (.1,1.55) -- (.9,1.55) -- (.9,1.95) -- (.1,1.95) -- cycle;
      % \filldraw [blue] (.1,2.05) -- (.9,2.05) -- (.9,2.45) -- (.1,2.45) -- cycle;
      % \filldraw [red] (.1,2.55) -- (.9,2.55) -- (.9,2.95) -- (.1,2.95) -- cycle;
      % ellipsis
      \filldraw [fill=black] (.5,2.2) circle (.5pt);
      \filldraw [fill=black] (.5,2.5) circle (.5pt);
      \filldraw [fill=black] (.5,2.8) circle (.5pt);
    \end{tikzpicture}
    \caption{The resulting interleaved vector.}
    \label{fig:mixedreorder_inner_vec}
  \end{subfigure}
  \caption{
    A possible data layout transformation for a `mixed' system permitted by \pyop3.
    The entries $V_0$ and $V_1$ represent the spaces of the mixed system and the ``Points" axis is representative of the mesh.
    Note that additional subaxes for, for example, nodes and \glspl{dof} would be permitted.
  }
  \label{fig:mixedreorder}
\end{figure}
\end{frame}

\begin{frame}{Partially-structured meshes: refined}
  \begin{figure}
    \centering
    \begin{subfigure}{\textwidth}
      \centering
      \begin{tikzpicture}[scale=.5]

        \tkzDefPoint(0,0){v0}
        \tkzDefShiftPoint[v0](60:4){v1}
        \tkzDefShiftPoint[v0](0:4){v2}
        \tkzDrawPolygon(v0,v1,v2)

        \tkzDefMidPoint(v0,v1) \tkzGetPoint{e0v0}
        \tkzDefMidPoint(v1,v2) \tkzGetPoint{e1v0}
        \tkzDefMidPoint(v2,v0) \tkzGetPoint{e2v0}

        \tkzDrawSegment(e0v0,e1v0)
        \tkzDrawSegment(e1v0,e2v0)
        \tkzDrawSegment(e2v0,e0v0)

        \tkzDefMidPoint(e1v0,v2) \tkzGetPoint{e1e1v0}
        \tkzDefMidPoint(v2,e2v0) \tkzGetPoint{e2e0v0}
        \tkzDefMidPoint(e1v0,e2v0) \tkzGetPoint{c0e2v0}

        \tkzDrawSegment(e1e1v0,e2e0v0)
        \tkzDrawSegment(e2e0v0,c0e2v0)
        \tkzDrawSegment(c0e2v0,e1e1v0)

        % patch
        \begin{pgfonlayer}{background}
          % find points by bisecting the angles
          \tkzDefShiftPoint[e0v0](-30:0.15){e0v0inner}
          \tkzDefShiftPoint[e1v0](-120:0.1){e1v0inner}
          \tkzDefShiftPoint[e1e1v0](150:0.15){e1e1v0inner}
          \tkzDefShiftPoint[c0e2v0](120:0.1){c0e2v0inner}
          \tkzDefShiftPoint[e2v0](90:0.15){e2v0inner}

          % source: https://tikz.dev/base-paths#sec-102.12
          \pgfsetcornersarced{\pgfpoint{1mm}{1mm}}
          \filldraw[color=blue!20] (e0v0inner) -- (e1v0inner) -- (e1e1v0inner) -- (c0e2v0inner) -- (e2v0inner) -- cycle;
          \pgfsetcornersarced{\pgfpointorigin}
        \end{pgfonlayer}

        % add labels
        \tkzDefBarycentricPoint(e0v0=1,e1v0=1,e2v0=1) \tkzGetPoint{c0c1}
        \node [xshift=-2cm,yshift=.8cm] (c0c1label) at (c0c1) {$(c_1,c_0)$};
        \draw (c0c1label) -- (c0c1);

        \tkzDefMidPoint(e1v0,c0e2v0) \tkzGetPoint{c0e2}
        \node [xshift=2cm,yshift=.8cm] (c0e2label) at (c0e2) {$(e_2,e_0)$};
        \draw (c0e2label) -- (c0e2);

        \tkzDefBarycentricPoint(e1v0=1,e1e1v0=1,c0e2v0=1) \tkzGetPoint{c0c3c2}
        \node [xshift=2cm,yshift=-.2cm] (c0c3c2label) at (c0c3c2) {$(c_3,c_2)$};
        \draw (c0c3c2label) -- (c0c3c2);
      \end{tikzpicture}
      \caption{
        An example of a stencil - $\plexstar((e_2,e_0))$ - over a refined mesh.
        Note that the unrefined cell $(c_1,c_0)$ is still indexed with two indices.
        We say that it has been refined using the identity transformation.
      }
      \label{fig:refined_patch}
    \end{subfigure}

    \vspace{1em}

    % data layout for patch
    \begin{subfigure}{\textwidth}
      \centering
      \begin{tikzpicture}[y=-1cm,scale=.5]
        \begin{scope}[xshift=.5cm, yshift=0cm]
          \filldraw[draw=black, fill=white] (0,0) rectangle ++ (1,1);
          \filldraw[draw=black, fill=blue!20] (1,0) rectangle ++ (1,1);
          \filldraw[draw=black, fill=white] (2,0) rectangle ++ (1,1);
          \filldraw[draw=black, fill=blue!20] (3,0) rectangle ++ (1,1);
          \filldraw[draw=black, fill=white] (4,0) rectangle ++ (1,1);
          \filldraw[draw=black, fill=blue!20] (5,0) rectangle ++ (1,1);
          \filldraw[draw=black, fill=white] (6,0) rectangle ++ (1,1);
          \node[at={(.5,.5)}, ptlabel] {$c_0$};
          \node[at={(1.5,.5)}, ptlabel] {$c_3$};
          \node[at={(2.5,.5)}, ptlabel] {$e_0$};
          \node[at={(3.5,.5)}, ptlabel] {$c_1$};
          \node[at={(4.5,.5)}, ptlabel] {$c_2$};
          \node[at={(5.5,.5)}, ptlabel] {$e_2$};
          \node[at={(6.5,.5)}, ptlabel] {$e_1$};

          % \draw[->] (2.8,-1) .. controls ([yshift=-.4cm] 2.6,-1) and ([yshift=.6cm] 1,0) .. (.8,0);
          % \draw[->] (3,-1) .. controls ([yshift=-.6cm] 3,-1) and ([yshift=1cm] 3.5,0) .. (3.5,0);
          % \draw[->] (3.2,-1) .. controls ([yshift=-.4cm] 3,-1) and ([yshift=.6cm] 6,0) .. (6.2,0);
        \end{scope}

        \begin{scope}[xshift=0cm, yshift=-2cm]
          % c3
          \begin{scope}[xshift=-1cm]
            \fill[lightgray] (0,0) rectangle (4,1);
            \filldraw[draw=black, fill=white] (.5,0) rectangle ++ (1,1);
            \filldraw[draw=black, fill=blue!20] (1.5,0) rectangle ++ (1,1);
            \filldraw[draw=black, fill=white] (2.5,0) rectangle ++ (1,1);
            \node[at={(1,.5)}, ptlabel] {$c_1$};
            \node[at={(2,.5)}, ptlabel] {$c_2$};
            \node[at={(3,.5)}, ptlabel] {$c_3$};
            \draw (0,0) -- (4,0);
            \draw (0,1) -- (4,1);
            % \draw[->] (4,-1) .. controls ([yshift=-.7cm] 4,-1) and ([yshift=1cm] 2,0) .. (2,0);
          \end{scope}

          % c1c0
          \begin{scope}[xshift=3.5cm]
            \filldraw[draw=black, fill=blue!20] (0,0) rectangle ++ (1,1);
            \node[at={(.5,.5)}, ptlabel] {$c_0$};
          \end{scope}

          % e2
          \begin{scope}[xshift=5cm]
            \filldraw[draw=black, fill=blue!20] (0,0) rectangle ++ (1,1);
            \filldraw[draw=black, fill=white] (1,0) rectangle ++ (1,1);
            \filldraw[draw=black, fill=white] (2,0) rectangle ++ (1,1);
            \node[at={(.5,.5)}, ptlabel] {$e_0$};
            \node[at={(1.5,.5)}, ptlabel] {$e_1$};
            \node[at={(2.5,.5)}, ptlabel] {$v_0$};
            % \draw[->] (2,-1) .. controls ([yshift=-.7cm] 2,-1) and ([yshift=1cm] .5,0) .. (.5,0);
          \end{scope}
        \end{scope}

        \draw [densely dashed] ({1+.5},1) -- ({0-1},2);
        \draw [densely dashed] ({2+.5},1) -- ({4-1},2);

        \draw ({3+.5},1) -- ({0+3.5},2);
        \draw ({4+.5},1) -- ({1+3.5},2);

        \draw ({5+.5},1) -- ({0+5},2);
        \draw ({6+.5},1) -- ({3+5},2);
      \end{tikzpicture}
      \caption{
        Example data layout for the refined mesh shown above.
        Note that the base mesh in unstructured which is why the top axis is unordered.
      }
      \label{fig:refined_data}
    \end{subfigure}
    \caption{Example data layout and stencil for a refined mesh.}
    \label{fig:refined_patch_and_data}
  \end{figure}
\end{frame}

\begin{frame}[fragile]{PCPATCH}
  \begin{minted}[fontsize=\tiny]{python}
loop(v := mesh.vertices.index, [
  loop(p := star(v).index, [
    assemble_jacobian(dat1[closure(p)], dat2[closure(p)], "mat"),
    assemble_residual(dat3[closure(p)], "vec"),
  ]),
  solve_and_update("mat", "vec", dat4[v]),
])
\end{minted}

\end{frame}

\end{document}
